\documentclass[aspectratio=169,10pt,handout]{beamer}
\usepackage{tikz}
\usepackage{xcolor}
\usepackage{fontspec}
   % http://tex.stackexchange.com/questions/132873/microtype-tracking-disables-small-caps-in-lualatex
   \defaultfontfeatures{SmallCapsFeatures={Renderer=Basic},}
   % Serif
   \setmainfont[%
      Numbers=OldStyle,%
      Renderer=Basic%
      ]
      {Tex Gyre Pagella}

   \setsansfont[%
      Renderer=Basic%
      ]      
      {TeX Gyre Heros}

\title{There Is No Largest Prime Number}
%\date[ISPN ’80]{27th International Symposium of Prime Numbers}
\author[Euclid]{Euclid of Alexandria \texttt{euclid@alexandria.edu}}
%\logo{\includegraphics[height=1cm]{UOW_logo_white.png}}

\usetheme[themecolor=red,themelayout=minimal]{uow}

\begin{document}

   \maketitle

\begin{frame} 
\frametitle{There Is No Largest Prime Number} 
\begin{theorem}
There is no largest prime number. 
\end{theorem} 
\begin{enumerate} 
\item Suppose $p$ were the largest prime number. 
\item Let $q$ be the product of the first $p$ numbers. 
\item Then $q+1$ is not divisible by any of them. 
\item But $q + 1$ is greater than $1$, thus divisible by some prime
number not in the first $p$ numbers.
\end{enumerate}
\end{frame}
% \section{Some example section}
% \begin{frame}[t]
%    \begin{tikzpicture}[overlay]
%       \draw[ultra thick, fill=UOWdarkred] (2,-2cm) circle (1cm);
%    \end{tikzpicture}
% \end{frame}
% 
\end{document}
